\documentclass[11pt, letterpaper, titlepage]{article}
\usepackage[utf8]{inputenc}
\usepackage{arsclassica}
\usepackage{geometry}
\usepackage{xcolor}
\usepackage{hyperref}

\geometry{a4paper,top=3cm,bottom=3cm,left=2cm,right=2cm,%
heightrounded,bindingoffset=5mm}


\begin{document}
\title{JDBC}
\author{Leonardo De Boni}
\maketitle
\tableofcontents
\newpage

\section{Ottenere una connessione}

\subsection{Import}
La classe da importare è:\par
\textcolor{brown}{java.io.Connection}

\subsection{Aprire una connessione}
Esistono vari metodi che restituiscono una connessione:
\begin{itemize}
    \item \textcolor{orange}{DriverManager}.\textcolor{red}{getConnection}(\textcolor{orange}{String} \textcolor{blue}{url})
    \item \textcolor{orange}{DriverManager}.\textcolor{red}{getConnection}(\textcolor{orange}{String} \textcolor{blue}{url}, \textcolor{orange}{String} \textcolor{blue}{user},\textcolor{orange}{String} \textcolor{blue}{pass})
    \item \textcolor{orange}{DriverManager}.\textcolor{red}{getConnection}(\textcolor{orange}{String} \textcolor{blue}{url}, \textcolor{orange}{Properties} \textcolor{cyan}{properties})
\end{itemize}
Formato dell'url:\par
\textcolor{blue}{jdbc:}\textcolor{yellow}{subprotocol:}\textcolor{green}{//host}\textcolor{cyan}{:porta}\textcolor{red}{/database}\\
Eventualmente l'url può contenere anche username e password e nel nostro caso l'url per MariaDb diventa:\par
\textcolor{blue}{jdbc:}\textcolor{yellow}{mariadb:}\textcolor{green}{//localhost}\textcolor{cyan}{:3306}\textcolor{red}{/database}?user=root\&password=myPassword\\
Connessione:\par
\textcolor{orange}{Connection} \textcolor{blue}{cn} = \textcolor{orange}{DriverManager}.\textcolor{red}{getConnection}(\textcolor{orange}{String} \textcolor{blue}{url});

\subsection{Chiudere una connessione}
Dopo avere eseguito le operazioni che mi interessano sulla connessione devo \textbf{sempre} chiudere la connessione:\par
\textcolor{blue}{cn}.\textcolor{red}{close}();

\section{Eccezioni}
Le connessioni sono delicate e possono generare errori e possono essere gestite tramite il costrutto try with resources:\par
\textcolor{magenta}{try} \textcolor{green}{(...dichiarazione oggetti autoclosable...)} \textcolor{magenta}{\{}\par
...\par
\textcolor{magenta}{\}}\\
dove all'interno delle parentesi tonde posso instanziare un oggetto che implementa \textbf{AutoClosable}
come ad esempio nel nostro caso è \textbf{Connection}.

\section{Statement}
\subsection{Creare uno statement}
Dopo aver creato una connessione come visto prima possono creare uno Statement a partire da essa:\par
\textcolor{orange}{Statement} \textcolor{blue}{st} = \textcolor{blue}{cn}.\textcolor{red}{createStatement}();

\subsection{Eseguire codice sql}
Per eseguire codice sql posso usare:\par
\textcolor{blue}{st}.\textcolor{red}{executeUpdate}(\textcolor{orange}{String} \textcolor{blue}{sql});\\
che posso usare per operazioni DML + DDL (insert, update, delete, drop, create, …)
e restituisce un \textbf{int} corrispondente al numero di righe modificate\par
\textcolor{blue}{st}.\textcolor{red}{executeQuery}(\textcolor{orange}{String}\textcolor{blue}{sql});\\
che posso usare per operazioni DQL (select) e retituisce un oggetto di tipo \textbf{ResultSet}.

\subsection{Eseguire codice sql generico}
Quando non so che tipo di query devo eseguire (es: di update o ricerca) posso usare il metodo:\par
\textcolor{blue}{st}.\textcolor{red}{execute}();\\
che ritorna un \textbf{boolean} equivalente a:
\begin{itemize}
    \item \textcolor{violet}{true} se il primo risultato è un \textbf{ResultSet} che posso ottenere
          con \textcolor{blue}{st}.\textcolor{red}{getResultSet}()
    \item \textcolor{violet}{false} se il primo risultato è il numero delle righe modificate che posso ottenere
          con \textcolor{blue}{st}.\textcolor{red}{getUpdateCount}()
\end{itemize}
Per controllare se ci sono altri risultati uso \textcolor{blue}{st}.\textcolor{red}{getMoreResults}().
So di avere finito i risultati perchè l'ultimo result sarà un count dal valore di -1.
\href{https://stackoverflow.com/questions/52700237/executequery-vs-getresultset-in-java#:~:text=boolean%20isResultSet%20%3D%20statement.execute(sql)%3B}{Codice di esempio}.

\section{ResultSet}
Quando ottengo un \textbf{ResultSet} posso muovermi su esso con un cursore che partirà dalla riga 0 (inesistente).
Quindi ho vari modi per muovere il cursore ed ottenere i dati relativi alla sua posizione.

\subsection{Estrarre dati dal ResultSet}
Per iterare sul \textbf{ResultSet} ho a disposizione il seguente metodo:\par
\textcolor{orange}{boolean} \textcolor{red}{next}();\\
Esempio:\par
\textcolor{magenta}{while} ( \textcolor{blue}{rs}.\textcolor{red}{next}() ) \textcolor{magenta}{\{}\par
...\par
\textcolor{magenta}{\}}\\
Esistono getters per ottenere i campi. Per esempio per ottenere un campo String ho varie
opzioni:
\begin{itemize}
    \item \textcolor{blue}{rs}.\textcolor{red}{getString}(\textcolor{purple}{"nomeCampo"})
    \item \textcolor{blue}{rs}.\textcolor{red}{getString}(\textcolor{purple}{1})
\end{itemize}
Il primo funziona in base al nome del campo che voglio ottenere il secondo in base alla
posizione del campo(sconsigliato).

\subsection{Movimenti avanzati sul ResultSet}
Oltre a \textcolor{red}{next}() ho a disposizione altri metodi per muovermi nel \textbf{ResultSet}:
\begin{itemize}
    \item \textcolor{red}{previous}()
    \item \textcolor{red}{first}()
    \item \textcolor{red}{last}()
    \item \textcolor{red}{absolute}(\textcolor{blue}{numero})
    \item \textcolor{red}{relative}(\textcolor{blue}{numero})
\end{itemize}

\subsection{Concorrenza}
//fare

\section{ResultSetMetaData}
//fare

\section{PreparedStatement}

\subsection{Creare un PreparedStatement}
Il \textbf{PreparedStatement} serve per scrivere una query parametrica, ovvero una query che
al posto dei dati che devo personalizzare ogni volta metto dei placeholder. Posso
creare un \textbf{PreparedStatement} in vari modi:
\begin{itemize}
    \item \textcolor{blue}{cn}.\textcolor{red}{prepareStatement}({\textcolor{blue}{sql}})
    \item \textcolor{blue}{cn}.\textcolor{red}{prepareStatement}({\textcolor{blue}{sql},\textcolor{orange}{int}}): INSERT
    \item \textcolor{blue}{cn}.\textcolor{red}{prepareStatement}({\textcolor{blue}{sql},\textcolor{orange}{int[]}}): INSERT
    \item \textcolor{blue}{cn}.\textcolor{red}{prepareStatement}({\textcolor{blue}{sql},\textcolor{orange}{String[]}}): INSERT
    \item \textcolor{blue}{cn}.\textcolor{red}{prepareStatement}({\textcolor{blue}{sql},\textcolor{orange}{int},\textcolor{orange}{int}})
\end{itemize}
Esempio di query sql parametrica:\par
\textcolor{purple}{INSERT INTO autori (id, cognome, nome) VALUES (?,?,?)}\\
Da notare l'utilizzo del carattere \textcolor{blue}{?} come placeholder. Quindi il
codice completo per creare un \textbf{PreparedStatement} può essere:\par
\textcolor{orange}{String} \textcolor{blue}{sql} = \textcolor{purple}{"INSERT INTO autori (id, cognome, nome) VALUES (?,?,?)"};\par
\textcolor{orange}{PreparedStatement} \textcolor{blue}{ps} = \textcolor{blue}{cn}.\textcolor{red}{prepareStatement}(\textcolor{blue}{sql});\\
Ora come faccio a impostare un valore arbitrario al posto dei placeholder?

\subsection{Operare su PreparedStatement}
Per sostituire i placeholder con valori arbitrari ho a disposizione vari setters che
accettano 2 argomenti di cui il primo contiene la posizione del placeholder che voglio
sostituire e il secondo il valore che voglio inserire. Esempio:
\begin{itemize}
    \item \textcolor{blue}{ps}.\textcolor{red}{setInt}(\textcolor{blue}{posizione},\textcolor{blue}{valore});
    \item \textcolor{blue}{ps}.\textcolor{red}{setString}(\textcolor{blue}{posizione},\textcolor{blue}{valore});
\end{itemize}

\subsection{Eseguire un PreparedStatement}
Una volta inseriti i dati al posto dei placeholder non ci resta che eseguire il
\textbf{PreparedStatement} in modo tale da operare sul database:\par
\begin{itemize}
    \item \textcolor{blue}{ps}.\textcolor{red}{executeUpdate}(); se applico una modifica
    \item \textcolor{blue}{ps}.\textcolor{red}{executeQuery}(); se interrogo il database per ottenere un \textbf{ResultSet}
\end{itemize}

\section{Query multiple}
Supponiamo di voler aggiungere più query a uno \textbf{Statement} oppure a uno \textbf{PreparedStatement}. In questo caso ci viene in
aiuto il metodo \textcolor{blue}{st}.\textcolor{red}{addBatch}(\textcolor{orange}{String} \textcolor{blue}{sql}) e
\textcolor{blue}{ps}.\textcolor{red}{addBatch}(). Il primo caso si spiega da solo il secondo invece si usa dopo aver impostato
il valore dei placeholder e quindi si può procedere a impostare di nuovo il valore dei placeholder in loop. Quando ho finito di aggiungere
query semplicemente chiamo il metodo \textcolor{red}{executeBatch}(). \href{https://www.baeldung.com/jdbc-batch-processing}{Esempi}

\section{Transazioni}
La transazione è un elenco di operazioni che vengono eseguite o \textbf{tutte o nessuna}.
Di default la connessione è in modalità auto commit e in automatico dopo ogni operazione viene applicata al database
in automatico. Posso disabilitare questa modalità e decidere io quando applicare le modifiche con:\par
\textcolor{blue}{cn}.\textcolor{red}{setAutoCommit}(\textcolor{purple}{false});\\
Per applicare le modifiche al database devo quindi usare:\par
\textcolor{blue}{cn}.\textcolor{red}{commit}();\\
Invece se voglio annullare le modifiche che ancora non ho applicato pur mantenendo la connessione uso:\par
\textcolor{blue}{cn}.\textcolor{red}{rollback}();

\section{ORM (Mapping)}
Gli \textbf{ORM} (Object-relational mapping) nascondono l'imlpementazione delle operazioni con il database permettendoci di operare
direttamente sugli oggetti. Traduzione: L'ORM si occupa del codice sql noi useremo solo metodi su oggetti per operare con
il database.

\subsection{DAO}
I \textbf{DAO} (Data Access Object) sono interfacce che espongono metodi che operano sul database.
L'impementazione di queste interfacce contengono quindi la logica JDBC.
Formalmente si possono riconoscere perchè il loro nome è composto dal nome dell'entità a cui si
riferiscono e DAO. es: Autore diventa AutoreDAO. In genere ogni tabella ha la sua classe corrispondente
(tabelle connessioni molti a molti escluse).

\end{document}